\documentclass{article}
\usepackage[utf8]{inputenc}
\usepackage[russian]{babel}
\usepackage{amsmath}
\usepackage{amssymb}
\usepackage{xcolor}

\title{Пример научной статьи для перевода}
\author{И. Иванов}
\date{\today}

\begin{document}

\maketitle

\begin{abstract}
В данной работе рассматриваются основные свойства гильбертовых пространств и применение операторной теории к задачам квантовой механики. Мы доказываем несколько важных теорем о спектральном разложении самосопряженных операторов.
\end{abstract}

\section{Введение}

Пусть $H$ --- гильбертово пространство со скалярным произведением $\langle \cdot, \cdot \rangle$. Рассмотрим линейный оператор $A: H \to H$. Говорят, что $\lambda \in \mathbb{C}$ является собственным значением оператора $A$, если существует ненулевой вектор $v \in H$ такой, что
\begin{equation}
A v = \lambda v.
\end{equation}

Теория операторов в гильбертовых пространствах имеет многочисленные применения в квантовой механике, где наблюдаемым величинам соответствуют самосопряженные операторы.

\section{Основные определения}

\subsection{Гильбертовы пространства}

Гильбертово пространство --- это полное векторное пространство со скалярным произведением. Норма в таком пространстве определяется через скалярное произведение:
\begin{equation}
\|v\| = \sqrt{\langle v, v \rangle}.
\end{equation}

Классическим примером является пространство $L^2(\mathbb{R})$ квадратично интегрируемых функций.

\subsection{Самосопряженные операторы}

Оператор $A$ называется самосопряженным, если $A = A^*$, где $A^*$ --- сопряженный оператор. Для самосопряженного оператора выполнено:
\[
\langle Av, w \rangle = \langle v, Aw \rangle
\]
для всех $v, w$ из области определения.

\section{Спектральная теорема}

Пусть $A$ --- компактный самосопряженный оператор в гильбертовом пространстве $H$. Тогда существует ортонормированный базис $\{e_n\}_{n=1}^{\infty}$ из собственных векторов оператора $A$ с собственными значениями $\{\lambda_n\}_{n=1}^{\infty}$, причем $\lambda_n \to 0$ при $n \to \infty$.

Это можно записать в виде спектрального разложения:
\begin{equation}
A = \sum_{n=1}^{\infty} \lambda_n \langle \cdot, e_n \rangle e_n.
\end{equation}

\section{Применения в квантовой механике}

В квантовой механике состояние системы описывается вектором $|\psi\rangle$ в гильбертовом пространстве. Гамильтониан системы $\hat{H}$ является самосопряженным оператором, а его собственные значения соответствуют возможным значениям энергии.

Уравнение Шрёдингера имеет вид:
\begin{equation}
i\hbar \frac{\partial}{\partial t}|\psi(t)\rangle = \hat{H}|\psi(t)\rangle.
\end{equation}

Здесь $\hbar$ --- приведенная постоянная Планка, а $i$ --- мнимая единица.

\section{Заключение}

Мы рассмотрели основы теории операторов в гильбертовых пространствах. Спектральная теорема является фундаментальным результатом, который находит широкое применение в различных областях математики и физики, особенно в квантовой механике.

\end{document}
